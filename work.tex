\documentclass[bachelor, och, pract]{SCWorks}
% параметр - тип обучения - одно из значений:
%    spec     - специальность
%    bachelor - бакалавриат (по умолчанию)
%    master   - магистратура
% параметр - форма обучения - одно из значений:
%    och   - очное (по умолчанию)
%    zaoch - заочное
% параметр - тип работы - одно из значений:
%    referat    - реферат
%    coursework - курсовая работа (по умолчанию)
%    diploma    - дипломная работа
%    pract      - отчет по практике
%    pract      - отчет о научно-исследовательской работе
%    autoref    - автореферат выпускной работы
%    assignment - задание на выпускную квалификационную работу
%    review     - отзыв руководителя
%    critique   - рецензия на выпускную работу
% параметр - включение шрифта
%    times    - включение шрифта Times New Roman (если установлен)
%               по умолчанию выключен
\usepackage[T2A]{fontenc}
\usepackage[utf8]{inputenc}
\usepackage{graphicx}

\usepackage{listings}
\usepackage{xcolor}
\lstset { %
    language=C++,
    %backgroundcolor=\color{black!5}, % set backgroundcolor
    %basicstyle=\footnotesize,% basic font setting
    basicstyle=\small
}

\usepackage[sort,compress]{cite}
\usepackage{amsmath}
\usepackage{amssymb}
\usepackage{amsthm}
\usepackage{fancyvrb}
\usepackage{longtable}
\usepackage{array}
\usepackage[english,russian]{babel}
\usepackage{minted}
% Используется автором репозитория
%\usemintedstyle{xcode}
% Этот пакет включает в себя аналогичный Times New Roman шрифт.
% Необходим для успешной компиляции для UNIX-систем ввиду отсутствия TNR в нем.
% Можно использовать и для Windows.
\usepackage{tempora}


\usepackage[colorlinks=false]{hyperref}

\usepackage{verbatim}


\newcommand{\eqdef}{\stackrel {\rm def}{=}}

\newtheorem{lem}{Лемма}

% % При использовании biblatex вместо bibtex
%\usepackage[style=gost-numeric]{biblatex}
%\addbibresource{thesis.bib}

\begin{document}

% Кафедра (в родительном падеже)
\chair{информатики и программирования}

% Тема работы
\title{Разработка древовидной иерархии модулей управления компонентами AltLinux}

% Курс
\course{3}

% Группа
\group{341}

% Факультет (в родительном падеже) (по умолчанию "факультета КНиИТ")
%\department{факультета КНиИТ}

% Специальность/направление код - наименование
%\napravlenie{02.03.02 "--- Фундаментальная информатика и информационные технологии}
\napravlenie{02.03.03 "--- Математическое обеспечение и администрирование информационных систем}
%\napravlenie{09.03.01 "--- Информатика и вычислительная техника}
%\napravlenie{09.03.04 "--- Программная инженерия}
%\napravlenie{10.05.01 "--- Компьютерная безопасность}

% Для студентки. Для работы студента следующая команда не нужна.
%\studenttitle{Студентки}

% Фамилия, имя, отчество в родительном падеже
\author{Шарова Кирилла Владимировича}

% Заведующий кафедрой
\chtitle{доцент, к.\,ф.-м.\,н.} % степень, звание
\chname{М.\,В.\,Огнева}

%Научный руководитель (для реферата преподаватель проверяющий работу)
\satitle{доцент, к.\,ф.-м.\,н.} %должность, степень, звание
\saname{? ?.?.}

% Руководитель практики от организации (только для практики,
% для остальных типов работ не используется)
\patitle{Заместитель генерального директора}
\paname{Е.\,А.\,Синельников}

% Семестр (только для практики, для остальных
% типов работ не используется)
\term{6}

% Наименование практики (только для практики, для остальных
% типов работ не используется)
\practtype{производственная}

% Продолжительность практики (количество недель) (только для практики,
% для остальных типов работ не используется)
\duration{4}

% Даты начала и окончания практики (только для практики, для остальных
% типов работ не используется)
\practStart{22.06.2024}
\practFinish{19.07.2024}

% Год выполнения отчета
\date{2024}

\maketitle

% Включение нумерации рисунков, формул и таблиц по разделам
% (по умолчанию - нумерация сквозная)
% (допускается оба вида нумерации)
%\secNumbering


\tableofcontents

% Раздел "Обозначения и сокращения". Может отсутствовать в работе
% \abbreviations
% \begin{description}
%     \item ... "--- ...
%     \item ... "--- ...
% \end{description}

% Раздел "Определения". Может отсутствовать в работе
%\definitions

% Раздел "Определения, обозначения и сокращения". Может отсутствовать в работе.
% Если присутствует, то заменяет собой разделы "Обозначения и сокращения" и "Определения"
%\defabbr


% Раздел "Введение"

\intro

AltLinux — это семейство российских дистрибутивов Linux, разрабатываемых компанией «Базальт СПО».
Они предназначены для использования в государственных учреждениях, образовательных организациях и других структурах, где требуется надёжное и безопасное программное обеспечение.

Одной из особенностей AltLinux является его ориентация на безопасность и защиту информации.
В дистрибутивы включены инструменты для шифрования данных, защиты от вирусов и несанкционированного доступа.
Также AltLinux поддерживает работу с отечественными криптографическими алгоритмами и средствами аутентификации.

Компания «Базальт СПО» предоставляет техническую поддержку и обновления для AltLinux, что обеспечивает его надёжность и стабильность.
Разработчики также проводят обучение и консультации по использованию системы, что способствует её распространению и внедрению.

В целом, AltLinux представляет собой надёжную и безопасную операционную систему, которая может быть использована в различных областях деятельности.
Она позволяет снизить зависимость от иностранных поставщиков программного обеспечения и обеспечить безопасность информационных систем\cite{MSDN}, % ссылка-заглушка пока что
что соответствует текущей государственной политике импортозамещения.

Целью работы является разработка древовидной иерархии модулей управления компонентами операционной системы AltLinux.

Поставленная цель определила следующие задачи:
\begin{itemize}
    \item Научиться собирать RPM пакеты инструментами AltLinux.
    \item Начать прохождение процедуры Join.
    \item Научиться разрабатывать и пакетировать приложения на C++ и Qt5.
    \item Изучить систему межпроцессного взаимодействия D-Bus.
    \item Реализовать древовидную иерархию компонентов в alterator-application-components.
\end{itemize}

\section{Раздел}

Раздел 1

\section{Раздел}

\subsection{Подраздел}

    Подраздел 2.1

\subsection{Подраздел}

    Подраздел 2.2

\newpage

% Раздел "Заключение"
\conclusion
Заключение.

Показано, как можно оформить документ в соответствии:
\begin{itemize}
    \item с правилами оформления курсовых и выпускных квалификационных работ, принятых в Саратовском государственном университете в 2012 году;
    \item с правилами оформления титульного листа отчета о прохождении практики в соответствии со стандартом.
\end{itemize}


%Библиографический список, составленный вручную, без использования BibTeX
%
%\begin{thebibliography}{99}
%  \bibitem{Ione} Источник 1.
%  \bibitem{Itwo} Источник 2
%\end{thebibliography}

%Библиографический список, составленный с помощью BibTeX
%
\bibliographystyle{gost780uv}
\bibliography{thesis}

% Окончание основного документа и начало приложений
% Каждая последующая секция документа будет являться приложением
\appendix

\section{Какое-то приложение}

\end{document}
