\documentclass[bachelor, och, pract]{SCWorks}
% параметр - тип обучения - одно из значений:
%    spec     - специальность
%    bachelor - бакалавриат (по умолчанию)
%    master   - магистратура
% параметр - форма обучения - одно из значений:
%    och   - очное (по умолчанию)
%    zaoch - заочное
% параметр - тип работы - одно из значений:
%    referat    - реферат
%    coursework - курсовая работа (по умолчанию)
%    diploma    - дипломная работа
%    pract      - отчет по практике
%    pract      - отчет о научно-исследовательской работе
%    autoref    - автореферат выпускной работы
%    assignment - задание на выпускную квалификационную работу
%    review     - отзыв руководителя
%    critique   - рецензия на выпускную работу
% параметр - включение шрифта
%    times    - включение шрифта Times New Roman (если установлен)
%               по умолчанию выключен
\usepackage[T2A]{fontenc}
\usepackage[utf8]{inputenc}
\usepackage{graphicx}

\usepackage{listings}
\usepackage{xcolor}
\lstset { %
    language=C++,
    %backgroundcolor=\color{black!5}, % set backgroundcolor
    %basicstyle=\footnotesize,% basic font setting
    basicstyle=\small
}

\usepackage[sort,compress]{cite}
\usepackage{amsmath}
\usepackage{amssymb}
\usepackage{amsthm}
\usepackage{fancyvrb}
\usepackage{longtable}
\usepackage{array}
\usepackage[english,russian]{babel}
\usepackage{minted}
% Используется автором репозитория
%\usemintedstyle{xcode}
% Этот пакет включает в себя аналогичный Times New Roman шрифт.
% Необходим для успешной компиляции для UNIX-систем ввиду отсутствия TNR в нем.
% Можно использовать и для Windows.
\usepackage{tempora}


\usepackage[colorlinks=false]{hyperref}

\usepackage{verbatim}


\newcommand{\eqdef}{\stackrel {\rm def}{=}}

\newtheorem{lem}{Лемма}

% % При использовании biblatex вместо bibtex
%\usepackage[style=gost-numeric]{biblatex}
%\addbibresource{thesis.bib}

\begin{document}

% Кафедра (в родительном падеже)
\chair{информатики и программирования}

% Тема работы
\title{???}

% Курс
\course{3}

% Группа
\group{341}

% Факультет (в родительном падеже) (по умолчанию "факультета КНиИТ")
%\department{факультета КНиИТ}

% Специальность/направление код - наименование
%\napravlenie{02.03.02 "--- Фундаментальная информатика и информационные технологии}
\napravlenie{02.03.03 "--- Математическое обеспечение и администрирование информационных систем}
%\napravlenie{09.03.01 "--- Информатика и вычислительная техника}
%\napravlenie{09.03.04 "--- Программная инженерия}
%\napravlenie{10.05.01 "--- Компьютерная безопасность}

% Для студентки. Для работы студента следующая команда не нужна.
%\studenttitle{Студентки}

% Фамилия, имя, отчество в родительном падеже
\author{Шарова Кирилла Владимировича}

% Заведующий кафедрой
\chtitle{доцент, к.\,ф.-м.\,н.} % степень, звание
\chname{М.\,В.\,Огнева}

%Научный руководитель (для реферата преподаватель проверяющий работу)
\satitle{доцент, к.\,ф.-м.\,н.} %должность, степень, звание
\saname{? ?.?.}

% Руководитель практики от организации (только для практики,
% для остальных типов работ не используется)
\patitle{ }
\paname{?.\,?.\,?}

% Семестр (только для практики, для остальных
% типов работ не используется)
\term{6}

% Наименование практики (только для практики, для остальных
% типов работ не используется)
\practtype{производственная}

% Продолжительность практики (количество недель) (только для практики,
% для остальных типов работ не используется)
\duration{4}

% Даты начала и окончания практики (только для практики, для остальных
% типов работ не используется)
\practStart{22.06.2024}
\practFinish{18.07.2024}

% Год выполнения отчета
\date{2024}

\maketitle

% Включение нумерации рисунков, формул и таблиц по разделам
% (по умолчанию - нумерация сквозная)
% (допускается оба вида нумерации)
%\secNumbering


\tableofcontents

% Раздел "Обозначения и сокращения". Может отсутствовать в работе
% \abbreviations
% \begin{description}
%     \item ... "--- ...
%     \item ... "--- ...
% \end{description}

% Раздел "Определения". Может отсутствовать в работе
%\definitions

% Раздел "Определения, обозначения и сокращения". Может отсутствовать в работе.
% Если присутствует, то заменяет собой разделы "Обозначения и сокращения" и "Определения"
%\defabbr


% Раздел "Введение"

\intro
Целью настоящей работы является создание примера оформления студенческой работы средствами системы \LaTeX.

Поставлена задача оформить документ в соответствии:
\begin{itemize}
    \item со стандартом СТО 1.04.01-2012 Порядком выполнения, структурой и правилами оформления курсовых работ (проектов)
    и выпускных квалификационных работ, принятых в Саратовском государственном университете в 2012 году;
    \item с правилами оформления титульного листа отчета о прохождении практики в соответствии со стандартом СТО 1.01-2005.
\end{itemize}

Изложенный ниже текст не имеет особого смысла и приведен только для демонстрации оформления своих элементов\cite{MSDN}.

\section{Сумма функционального ряда}

Вычислить сумму ряда следующего функционального ряда:

$$ f(x) = x - \frac{x^3}{3} + \frac{x^5}{5} - \frac{x^7}{7} - ... =  \sum_{n=0}^{\infty} (-1)^k\frac{x^{2k+1}}{(2k+1)!}$$

Условие остановки вычислений: $|f_{n+1}(x)| < \epsilon = 10^{-3}$

\subsubsection{Код}

\begin{lstlisting}
void f(double x, double e)
{
    double S = x;
    double Fn =S;
    int i = 1;
    while (!abs(Fn) < e)
    {
        Fn *= ((-1) * x * x) / (2 * i * (2 * i + 1));
        S += Fn;
        i++;
    }
    
    std::cout << x << '\t' << S << '\t' << i <<'\n';
}
\end{lstlisting}

\subsubsection{Вывод}

Формат вывода:


\begin{table}[h]
    \centering
    \begin{tabular}{|c|c|c|c|}
         \hline$x_0$& $x_1$ & ... & $x_m$\\\hline
        $S_{n_0}(x_0)$ & $S_{n_1}(x_1)$ & ... & $S_{n_m}(x_m)$ \\\hline
        $n_0$ & $n_1$ & ... &$n_m$ \\\hline
    \end{tabular}
   % \caption{Caption}
   % \label{tab:my_label}
\end{table}


Полученный вывод:

\begin{table}[h]
    \centering
    \begin{tabular}{|c|c|c|c|c|}
         \hline-2 & -1 & 0 & 1 & 2\\\hline
          -0.909297 &  -0.841471 & 0 & 0.841471 & 0.909297 \\\hline
        103 & 90& 1 & 90 & 103 \\\hline
    \end{tabular}
   % \caption{Caption}
   % \label{tab:my_label}
\end{table}


\section{Интерполяция функций одного аргумента}

\subsection{Интерполяционный многочлен в общем виде}

Пусть известны значения функции f(x) в некотором наборе узловых точек:

\begin{table}[h]
    \centering
    \begin{tabular}{|c|c|c|c|c|}
         \hline x & -2 & -1 & 0 & 1\\\hline
          f(x) &  -8 &-1 & 0 & 1 \\\hline
    \end{tabular}
   % \caption{Caption}
   % \label{tab:my_label}
\end{table}

Тогда интерполяционный многочлен функции f(x) будет иметь вид:
$$P_3(x) = a_3x^3 + a_2x^2 + a_1x + a_0$$

Построить расширенную матрицу коэффициентов (РМК) относительно коэффициентов $a_i, i = 0...3$.

\subsubsection{Код}

\begin{lstlisting}
struct MatrixOfCoefficients
{
    MatrixOfCoefficients(std::vector<double> Xi, 
    std::vector<double> Fi)
    {
        for (size_t i = 0; i < Xi.size(); i++)
        {
            matrix.push_back(std::vector<double>());
            for (size_t j = 0; j < Fi.size(); ++j)
            {
                matrix[i].push_back(std::pow(Xi[i], 
                Xi.size() - j - 1));
            }
            matrix[i].push_back(Fi[i]);
        }
    }

    void print()
    {
        std::cout << "Matrix of Coefficient:\n";
        const size_t width = 10;
        std::cout << std::fixed << std::setprecision(3);
        for (size_t i = 0; i < matrix.size(); i++, 
        std::cout<<'\n')
            for (size_t j = 0; j < matrix[i].size(); ++j)
                std::cout << std::setw(width) 
                << matrix[i][j];
    }

private:
    std::vector<std::vector<double>> matrix;
};
\end{lstlisting}

\subsubsection{Вывод}

Полученная таблица:

\begin{table}[h]
    \centering
    \begin{tabular}{|c|c|c|c|c|}
\hline -8.000  &   4.000  &  -2.000  &   1.000  &  8.000\\\hline
    -1.000  &   1.000  &  -1.000  &   1.000  &  -1.000\\\hline
     0.000  &   0.000  &   0.000  &   1.000  &   0.000\\\hline
     1.000  &   1.000  &   1.000  &   1.000  &   1.000\\\hline
    \end{tabular}
   % \caption{Caption}
   % \label{tab:my_label}
\end{table}

\subsection{Интерполяционный многочлен в форме Лагранжа}

    Построить интерполяционный многочлен в форме Лагранжа для даных из предыдущего задания.

\subsubsection{Код}

\begin{lstlisting}
struct LagrangeTable
{
    LagrangeTable(std::vector<double> Xi, 
    std::vector<double> Fi)
    {
        this->Fi = Fi;
        this->Xi = Xi;
    }

    void Print()
    {
        std::cout << "Lagrange Polynom function's table:\n";
        const size_t width = 10;
        std::cout << std::fixed << std::setprecision(3) 
        << "Xi: ";
        for (size_t i = 0; i < Xi.size(); i++)
        {
            std::cout << std::setw(width) << Xi[i] << ' ';
            if (i != Xi.size() - 1)
            {
                std::cout << std::setw(width) 
                << (Xi[i] + Xi[i+1])/2. << ' ';
            }
        }
        std::cout << "\nFi: ";
        for (size_t i = 0; i < Fi.size(); i++)
        {
            std::cout << std::setw(width) << Fi[i] << ' ';
            if (i != Fi.size() - 1)
            {
                std::cout << std::setw(width) << 
                LagrangePolynom((Xi[i] + Xi[i + 1]) / 2., 
                Xi, Fi) << ' ';
            }
        }
        std::cout << "\n";
    }

private:
    std::vector<double> Xi;
    std::vector<double> Fi;
};

double LagrangePolynom(double x, std::vector<double> Xi, 
std::vector<double> Fi)
{
    double L = 0;
    double currentSum;
    for (int i = 0; i < Xi.size(); i++)
    {
        currentSum = Fi[i];
        for (int j = 0; j < Xi.size(); j++)
        {
            if (j != i)
            {
                currentSum *= (x - Xi[j]) / (Xi[i]-Xi[j]);
            }
        }
        L += currentSum;
    }
    return L;
}
\end{lstlisting}


\subsubsection{Вывод}

Полученный вывод:

\begin{table}[h]
    \centering
    \begin{tabular}{|c|c|c|c|c|c|c|c|}
         \hline x & -2 & -1.5 & -1 & -0.5 & 0 & 0.5 & 1\\\hline
          f(x) &  -8 & -3.375 & -1 & -0.125 & 0 & 0.125 & 1 \\\hline
    \end{tabular}
   % \caption{Caption}
   % \label{tab:my_label}
\end{table}


\subsection{Интерполяционный многочлен в форме Ньютона}

Построить интерполяционный многочлен в форме Ньютона по данным из предыдущего задания.

\subsubsection{Код}

\begin{lstlisting}
struct NewtonTable
{
    NewtonTable(std::vector<double> Fi, 
    std::vector<double> knots)
    {
        Knots = knots;
        for (size_t i = 0; i < Fi.size(); ++i)
        {
            SplitDifs.push_back(std::vector<double>());
        }
        for (size_t i = 0; i < Fi.size(); ++i)
        {
            SplitDifs[i].push_back(Fi[i]);
        }
        for (size_t i = 1; i < Fi.size(); i++)
        {
            for (size_t j = 0; j < Fi.size() - i; j++)
            {
                SplitDifs[j].push_back(
                (SplitDifs[j+1][i-1] - SplitDifs[j][i-1])/
                (knots[j+i] - knots[j]));
            }
        }
    }

    void print()
    {
        const size_t width = 10;
        std::cout << std::fixed << std::setprecision(3);
        for (size_t i=0; i<Knots.size(); i++, 
        std::cout<<'\n')
            for (size_t j = 0; j < Knots.size() - i; j++)
            {
                std::cout << std::setw(width) << 
                SplitDifs[i][j]<<' ';
            }
    }

    std::vector<std::vector<double>> GetSplitDifs() 
    { return SplitDifs; }

private:
    std::vector<std::vector<double>> SplitDifs;
    std::vector<double> Knots;
};


double NewtonPolynom(double x, std::vector<double> Xi,
std::vector<std::vector<double>> SplitDifs)
{
    double ans = 0;
    double temp;
    for (size_t i = 0; i < SplitDifs[0].size(); i++)
    {
        temp = SplitDifs[0][i];
        for (size_t j = 0; j < i; j++)
        {
            temp *= (x - Xi[j]);
        }
        ans += temp;
    }
    return ans;
}
\end{lstlisting}

\subsubsection{Вывод}

Таблица разделённых разностей:

\begin{table}[h]
    \centering
    \begin{tabular}{|c|c|c|c|}
\hline-8.000  &    7.000  &   -3.000  &    1.000\\\hline
    -1.000  &    1.000   &   0.000   & \\\hline
     0.000  &    1.000 & & \\\hline
     1.000 & & & \\\hline
    \end{tabular}
   % \caption{Caption}
   % \label{tab:my_label}
\end{table}

Таблица значений интерполяционного многочлена в форме Ньютона:

\begin{table}[H]
    \centering
    \begin{tabular}{|c|c|c|c|c|c|c|c|}
         \hline x & -2 & -1.5 & -1 & -0.5 & 0 & 0.5 & 1\\\hline
          f(x) &  -8 & -3.375 & -1 & -0.125 & 0 & 0.125 & 1 \\\hline
    \end{tabular}
   % \caption{Caption}
   % \label{tab:my_label}
\end{table}



\subsection{Интерполяция с помощью кубических сплайнов}

Построить кусочно-непрерывную склейку кубических сплайнов для
функции f(x), заданной в предыдущих заданиях.

Вспомогательная система:

$$ 
\begin{cases}
a_1 = f_0 \\
a_2 = f_1 \\
a_3 = f_2 \\
a_1 + b_1h + c_1h^2 + d_1h^3 = f_1 \\
a_2 + b_2h + c_2h^2 + d_2h^3 = f_2 \\
a_3 + b_3h + c_3h^2 + d_3h^3 = f_3 \\
b_1 + 2c_1h + 3d_1h^2 = b_2 \\
b_2 + 2c_2h + 3d_2h^2 = b_3 \\
2c_1 + 6d_1h = 2c_2 \\
2c_2 + 6d_2h = 2c_3 \\
2c_3 + 6d_3h = 0 \\
2c_1 = 0
\end{cases} $$


\subsubsection{Код}

\begin{lstlisting}
struct GaussTable
{
public:
    GaussTable(double h, std::vector<double> Fi)
    {
    table = {{1,0,0,0,0,0,0,0,0,0,0,0, Fi[0]},
             {0,1,0,0,0,0,0,0,0,0,0,0, Fi[1]},
             {0,0,1,0,0,0,0,0,0,0,0,0, Fi[2]},
             {1,0,0,h,0,0,h * h,0,0,h * h * h,0,0,Fi[1]},
             {0,1,0,0,h,0,0,h * h,0,0,h * h * h,0,0,Fi[2]},
             {0,0,1,0,0,h,0,0,h * h,0,0,h * h * h,Fi[3]},
             {0,0,0,1,-1,0,2*h,0,0,3*h*h,0,0,0},
             {0,0,0,0,1,-1,0,2*h,0,0,3*h*h,0,0},
             {0,0,0,0,0,0,2,-2,0,6*h,0,0,0},
             {0,0,0,0,0,0,0,2,-2,0,6*h,0,0},
             {0,0,0,0,0,0,2,0,0,0,0,0,0},
             {0,0,0,0,0,0,0,0,2,0,0,6*h,0}
        };
    }
    void Print()
    {
        const size_t width = 8;
        std::cout << std::fixed << std::setprecision(3);
        for (size_t i = 0; i < table.size(); i++, 
        std::cout << '\n')
            for (size_t j = 0; j < table[0].size(); j++)
            {
                if (j == table[0].size()-1)
                    std::cout << std::setw(width) << '|';
                std::cout << std::setw(width) << table[i][j] 
                << ' ';
            }
    }

private:
    std::vector<std::vector<double>> table;
};
\end{lstlisting}


\subsubsection{Вывод}

При h = 1

\begin{table}[H]
    \centering
    \begin{tabular}{|c|c|c|c|c|c|c|c|c|c|c|c|c|}
\hline   1  &  0 &   0  &  0 &   0  &  0  &  0 &   0  &  0  &  0  &  0  &  0   &   -8\\\hline
   0  &  1  &  0 &   0  &  0   &  0   &  0  &   0   &  0  &  0  &  0  &  0   &-1 \\\hline
   0  &  0  &  1   &  0  &   0  &   0  & 0  &    0  &   0   &  0   &  0  &   0    &    0  \\\hline
   1  &  0 &   0  &   1  &   0  &   0  &   1  &   0    & 0   &  1   &  0  &   0     &   -1  \\\hline
   0 &   1  &  0  &   0  &   1  &   0   &  0  &   1   &  0  &   0  &   1  &   0   &       0\\\hline
   0  &  0  &  1   &  0   &  0   &  1   &  0  &   0   &  1   &  0   &  0   &  1   &     1 \\\hline
   0  &  0  &  0   &  1   & -1   &  0  &   2    & 0   &  0    & 3   &  0   &  0    &   0 \\\hline
   0 &   0 &   0   &  0   &  1  &  -1   &  0  &   2  &   0   &  0  &   3 &   0    &   0 \\\hline
   0  &  0   & 0   &  0  &   0  &   0   &  2   & -2   &  0  &  6   &  0   &  0  &    0 \\\hline
   0 &   0  &  0  &   0   &  0  &   0   &  0    & 2  &  -2    & 0   &  6  &   0  &     0 \\\hline
   0  &  0  &  0  &   0   &  0  &   0 0  &  2  &   0   &  0   &  0   &  0   &  0    &    0 \\\hline
   0  &  0   & 0  &   0   &  0  &   0   &  0   &  0  &   2  &  0   &  0  &   6    &    0 \\\hline
    \end{tabular}
   % \caption{Caption}
   % \label{tab:my_label}
\end{table}


\section{Численные методы решения СЛАУ}

\subsection{Метод Гаусса решения СЛАУ}

Здесь и далее в работе v = 5.

Имеем:
$$ A = 
\begin{pmatrix}
5 & 0.05 & 0.05 & 0.05 & 0.05 \\
0.06 & 6 & 0.06 & 0.06 & 0.06 \\
0.07 & 0.07 & 7 & 0.07 & 0.07 \\
0.08 & 0.08 & 0.08 & 8 & 0.08 \\
0.09 & 0.09 & 0.09 & 0.09 & 9 
\end{pmatrix}
$$

$$ B = A *
\begin{pmatrix}
5 \\
6 \\
7 \\
8 \\
9
\end{pmatrix}
$$

Необходимо вывести: 
\begin{enumerate}
    \item РМК (A|B)
    \item Результирующую матрицу прямого хода метода Гаусса
    \item Результирующий вектор ($x_1, x_2, ..., x_n$) обратного хода метода Гаусса
\end{enumerate}

\subsubsection{Код}

\begin{lstlisting}
void Matrix::printGaussAB(const Matrix& A, const Matrix& B)
{
    const size_t width = 10;
    std::cout << std::fixed << std::setprecision(3);
    for (size_t i = 0; i < A.getN(); i++)
    {
        for (size_t j = 0; j < A.getM(); j++)
        {
            std::cout << std::setw(width) 
            << A.mat[i][j] << ' ';
        }
        std::cout << std::setw(width) << "| " 
        << B.mat[i][0] << '\n';
    }
}


Matrix Matrix::merge(const Matrix& A, const Matrix& B)
{
    Matrix ans;
    ans.n = A.n;
    ans.m = A.m + B.m;
    for (int i = 0; i < A.n; i++)
    {
        ans.mat.push_back(std::vector<double>(ans.m));
    }
    for (int i = 0; i < A.n; i++)
    {
        for (int j = 0; j < A.m; j++)
        {
            ans.mat[i][j] = A.mat[i][j];
        }
        for (int j = 0; j < B.m; j++)
        {
            ans.mat[i][A.m + j] = B.mat[i][j];
        }
    }

    return ans;
}

Matrix Matrix::forward_gauss(const Matrix& A, 
const Matrix& B)
{
    Matrix ans = merge(A, B);
    for (int i = 0; i < ans.n; i++)
    {
        double koeff = 1, eps = 1e-6;
        if (std::abs(ans.mat[i][i]) > eps)
            koeff = ans.mat[i][i];
        else
        {
            bool isChanged = false;
            double maxEl = 0;
            int index = -1;
            for (int j = i + 1; j < ans.n; j++)
            {
                if (std::abs(std::abs(ans.mat[j][i]) - maxEl)
                > eps)
                {
                    maxEl = ans.mat[j][i];
                    index = j;
                }
            }
            if (index != -1)
            {
                std::swap(ans.mat[i], ans.mat[index]);
                isChanged = true;
                koeff = ans.mat[i][i];
            }
            if (!isChanged)
            {
                maxEl = 0;
                index = -1;
                for (int j = 0; j < ans.m; j++)
                {
                    if (std::abs(std::abs(ans.mat[i][j]) 
                    - maxEl) > eps)
                    {
                        maxEl = ans.mat[i][j];
                        index = j;
                    }
                }
                if (index != -1)
                {
                    for (int j = 0; j < ans.n; j++)
                        std::swap(ans.mat[j][i], 
                        ans.mat[j][index]);
                    isChanged = true;
                    koeff = ans.mat[i][i];
                }
            }
            if (!isChanged)
                continue;
        }
        for (int j = i; j < ans.m; j++)
        {
            ans.mat[i][j] = ans.mat[i][j] / koeff;
        }
        for (int j = i + 1; j < ans.n; j++)
        {
            
            for (int k = ans.m-1; k >= i; k--)
            {
                double diff = ans.mat[i][k] * ans.mat[j][i];
                ans.mat[j][k] -= diff;
            }
        }
    }


    return ans;
}

std::vector<double> Matrix::back_gauss(const Matrix& A)
{
    std::vector<double> Xn(A.getN());
    for (int i = A.getN() - 1; i>=0; i--)
    {
        Xn[i] = A.mat[i][A.getM()-1];
        for (int j = A.getM()-2; j > i; j--)
        {
            Xn[i] -= Xn[j] * A.mat[i][j];
        }
    }
    return Xn;
}
\end{lstlisting}

\subsubsection{Вывод}

1)

\begin{table}[h]
    \centering
    \begin{tabular}{|c|c|c|c|c|c|}
\hline 5.000  &   0.050  &    0.050   &   0.050  &  0.050 & 26.500 \\\hline
     0.060  & 6.000 &  0.060 &  0.060  &  0.060 & 37.740\\\hline
     0.070  &    0.070  &    7.000   &  0.070  &  0.070  & 50.960\\\hline
     0.080   &   0.080   &   0.080   &   8.000 &  0.080 & 66.160\\\hline
     0.090 &  0.090 &   0.090 &  0.090 &  9.000  & 83.340 \\\hline
    \end{tabular}
   % \caption{Caption}
   % \label{tab:my_label}
\end{table}

2)

\begin{table}[h]
    \centering
    \begin{tabular}{|c|c|c|c|c|c|}
\hline 1.000   & 0.010 &  0.010  &  0.010 &  0.010 & 5.300\\\hline
     0.000  & 1.000  &  0.010  & 0.010  &  0.010  &  6.238\\\hline
     0.000  & 0.000  & 1.000  &  0.010  &  0.010 &7.167\\\hline
     0.000  & 0.000  &0.000  &  1.000  &  0.010 & 8.087\\\hline
     0.000 &  0.000  &  0.000 &  0.000 &  1.000 & 9.000\\\hline
    \end{tabular}
   % \caption{Caption}
   % \label{tab:my_label}
\end{table}

3)
\begin{table}[h]
    \centering
    \begin{tabular}{|c|c|c|c|c|c|}
\hline 5.000 & 6.000 & 7.000 & 8.000 & 9.000\\\hline
    \end{tabular}
   % \caption{Caption}
   % \label{tab:my_label}
\end{table}


\subsection{Приложения метода Гаусса}

Используя метод Гаусса вычислить определитель и обратную матрицу для матрицы
$$A = \begin{pmatrix}
-9 & 1 & 3 & 4 \\
3 & 0 & -1 & 4 \\
-5 & 2 & 3 & 0 \\
4 & -1 & 2 & 6 \\
\end{pmatrix}$$


\subsubsection{Код}

\begin{lstlisting}
double Matrix::determinant(const Matrix& A)
{
    Matrix copy = A;
    int changes_count = 0;
    double determinant_koeff = 1;
    for (int i = 0; i < copy.n; i++)
    {
        double koeff = 1, eps = 1e-6;
        if (std::abs(copy.mat[i][i]) > eps)
            koeff = copy.mat[i][i];
        else
        {
            bool isChanged = false;
            double maxEl = 0;
            int index = -1;
            for (int j = i + 1; j < copy.n; j++)
            {
                if (std::abs(std::abs(copy.mat[j][i]) - 
                maxEl) > eps)
                {
                    maxEl = copy.mat[j][i];
                    index = j;
                }
            }
            if (index != -1)
            {
                std::swap(copy.mat[i], copy.mat[index]);
                isChanged = true;
                koeff = copy.mat[i][i];
            }
            if (!isChanged)
            {
                maxEl = 0;
                index = -1;
                for (int j = 0; j < copy.m; j++)
                {
                    if (std::abs(std::abs(copy.mat[i][j]) - 
                    maxEl) > eps)
                    {
                        maxEl = copy.mat[i][j];
                        index = j;
                    }
                }
                if (index != -1)
                {
                    for (int j = 0; j < copy.n; j++)
                        std::swap(copy.mat[j][i], 
                        copy.mat[j][index]);
                    isChanged = true;
                    koeff = copy.mat[i][i];
                }
            }
            if (isChanged)
                changes_count += 1;
            if (!isChanged)
                continue;
        }
        for (int j = i; j < copy.m; j++)
        {
            copy.mat[i][j] = copy.mat[i][j] / koeff;
        }
        determinant_koeff *= koeff;
        for (int j = i + 1; j < copy.n; j++)
        {

            for (int k = copy.m - 1; k >= i; k--)
            {
               double diff = copy.mat[i][k] * copy.mat[j][i];
               copy.mat[j][k] -= diff;
            }
        }
    }
    return std::pow(1, changes_count) * determinant_koeff;
}

Matrix Matrix::reverse_matrix(const Matrix& A)
{
    Matrix B = A;
    Matrix ans(A.n, A.m);
    std::vector<std::vector<double>> collumn;
    for (size_t i = 0; i < A.n; i++)
        collumn.push_back(std::vector<double>(1));
    for (size_t i = 0; i < A.m; i++)
    {
        collumn[i][0] = 1;
        Matrix temp = forward_gauss(B, collumn);
        std::vector<double> current_collumn = 
        back_gauss(temp);
        for (size_t j = 0; j < ans.n; j++)
        {
            ans.mat[j][i] = current_collumn[j];
        }
        collumn[i][0] = 0;
    }

    return ans;
}
\end{lstlisting}

\subsubsection{Вывод}

$\Delta$A = -272


$A^{-1} = \begin{pmatrix}
-0.059  &    0.169  &    0.066  &    0.074\\
    -0.412   &   0.059    &  0.588 &    -0.235\\
     0.176    &  0.243    &  0.051  &    0.279\\
     0.088    &  0.184   &  -0.037   &   0.015\\
\end{pmatrix}$


\subsection{Метод прогонки решения СЛАУ с трёхдиагональной матрицей
коэффициентов}

При помощи метода прогонки решить СЛАУ вида $A*x = B$, где
 $$A = \begin{pmatrix}
-5 & 0.05 & 0 & 0 & 0 \\
0.06 & -6 & 0.06 & 0 & 0 \\
0 & 0.07 & -7 & 0.07 & 0 \\
0 & 0 & 0.08 & -8 & 0.08 \\
0 & 0 & 0 & 0.09 & -9 
\end{pmatrix}$$

$$ B = A *
\begin{pmatrix}
5 \\
6 \\
7 \\
8 \\
9
\end{pmatrix}
$$

Необходимо вывести: 
\begin{enumerate}
    \item РМК (A|B)
    \item Результирующую матрицу прямого хода метода прогонки (коэффициенты P и Q)
    \item Результирующий вектор ($x_1, x_2, ..., x_n$) обратного хода метода прогонки
\end{enumerate}

\subsubsection{Код}

\begin{lstlisting}
std::vector<std::vector<double>> 
Matrix::three_diagonal_matrix(const Matrix& A, 
const std::vector<double>& d)
{
    std::vector<double> P(A.getN()- 1);
    std::vector<double> Q(A.getN());
    P[0] = -A[0][1] / A[0][0]; Q[0] = d[0]/A[0][0];
    for (int i = 1; i < A.getN() - 1; i++)
    {
        double det = -A[i][i] - A[i][i - 1] * P[i - 1];
        P[i] = A[i][i + 1] / det;
        Q[i] = (A[i][i - 1] * Q[i - 1] - d[i]) / det;
        
    }
    size_t size = A.getN();
    Q[A.getN() - 1] = (A[size - 1][size - 2] * 
    Q[size - 2] - d[size - 1]) / (-A[size - 1][size - 1] - 
    A[size - 1][size - 2] * P[size - 2]);

    size = Q.size();
    std::vector<double> Xn(size);
    Xn[size - 1] = Q[size - 1];
    for (int i = size-1; i > 0; --i)
    {
        Xn[i - 1] = P[i - 1] * Xn[i] + Q[i - 1];
    }

    return std::vector<std::vector<double>>{P, Q, Xn};
}    
\end{lstlisting}

\subsubsection{Вывод}

1)
\begin{table}[H]
    \centering
    \begin{tabular}{|c|c|c|c|c|c|}
\hline -5.000 & 0.050  & 0.000 & 0.000 &  0.000   &-24.700\\\hline
     0.060 & -6.000  &0.060 & 0.000 & 0.000 & -35.280\\\hline
     0.000 & 0.070  & -7.000  & 0.070& 0.000 & -48.020\\\hline
     0.000 & 0.000 &  0.080 & -8.000&  0.080  & -62.720\\\hline
     0.000 & 0.000  & 0.000 &  0.090&-9.000 & -80.280\\\hline
    \end{tabular}
   % \caption{Caption}
   % \label{tab:my_label}
\end{table}

2)
\begin{table}[H]
    \centering
    \begin{tabular}{|c|c|c|c|c|c|}
\hline P &0.01& 0.010001 &0.010001& 0.010001 &\\\hline
Q&4.94& 5.92999 &6.91999 &7.90999 &9 \\\hline
    \end{tabular}
   % \caption{Caption}
   % \label{tab:my_label}
\end{table}

3)
\begin{table}[H]
    \centering
    \begin{tabular}{|c|c|c|c|c|c|}
\hline 5&6&7&8&9 \\\hline
    \end{tabular}
   % \caption{Caption}
   % \label{tab:my_label}
\end{table}


\section{Численные методы решения дифференциальных уравнений}

\subsection{Метод Эйлера решения задачи Коши для обыкновенного
дифференциального уравнения (ОДУ) 1-го порядка}

Решить следующую задачу Коши методом Эйлера, усовершенствованным методом Эйлера, методом предварительного и корректирующего счёта:

$$ 
\begin{cases}
y'(x) = 2 * 5 * x + v * x^2 - y(x) \\
y(1) = 5
\end{cases} $$,

где $y_{correct}(x) = 5 * x^2$

Необходимо вывести:
\begin{enumerate}
    \item Таблицу результатов для метода Эйлера
    \item Таблицу результатов для усовершенствованного метода Эйлера
    \item Таблицу результатов для метода предварительного и корректирующего счёта
\end{enumerate}

Общий вид таблицы результатов (здесь и далее в работе):


где $y_{met}(x_i)$ - значение метода в точке $x_i$, $y_{correct}(x_i)$ - истинное значение функции y(x) в точке $x_i$, $e_i$ - погрешность метода в точке $x_i$.

\subsubsection{Код}
\begin{lstlisting}
std::vector<double> eiler_method(std::vector<double> Xk, 
double Y0, double h, std::function<double(double, double)> f)
{
    std::vector<double> Yk;
    Yk.push_back(Y0);
    for (size_t i = 1; i < Xk.size(); i++)
    {
        Yk.push_back(Yk[i - 1] + h * f(Xk[i - 1], 
        Yk[i - 1]));
    }

    return Yk;
}

std::vector<double> modernize_eiler_method
(std::vector<double> Xk, double Y0, double h, 
std::function<double(double, double)> f)
{
    std::vector<double> Yk;
    Yk.push_back(Y0);

    for (size_t i = 1; i < Xk.size(); i++)
    {
        Yk.push_back(Yk[i - 1] + h * f(Xk[i-1] + h/2, 
        Yk[i-1] + (h/2)* f(Xk[i-1], Yk[i-1])));
    }

    return Yk;
}


std::vector<double> correct_count_method
(std::vector<double> Xk, double Y0, double h,
std::function<double(double, double)> f)
{
    std::vector<double> Yk;
    Yk.push_back(Y0);

    for (size_t i = 1; i < Xk.size(); i++)
    {
        Yk.push_back(Yk[i - 1] + (h/2) * (f(Xk[i - 1], 
        Yk[i - 1]) + f(Xk[i], 
        Yk[i-1] + h * f(Xk[i-1], Yk[i-1]))));
    }

    return Yk;
}


void print_eiler_method(std::vector<double> Xk, 
std::vector<double> Yk, std::vector<double> Y)
{
    const size_t width = 10;
    std::cout << std::fixed << std::setprecision(3);
    for (size_t i = 0; i < Xk.size(); i++)
    {
        std::cout << std::setw(width) << Xk[i];
    }
    std::cout << "\n";
    for (size_t i = 0; i < Xk.size(); i++)
    {
        std::cout << std::setw(width) << Yk[i];
    }
    std::cout << "\n";
    for (size_t i = 0; i < Xk.size(); i++)
    {
        std::cout << std::setw(width) << Y[i];
    }
    std::cout << "\n";
    for (size_t i = 0; i < Yk.size(); i++)
    {
        std::cout << std::setw(width) << Yk[i] - Y[i];
    }
}
\end{lstlisting}


\subsubsection{Вывод}

1)

\begin{table}[H]
    \centering
    \begin{tabular}{|c|c|c|c|c|c|c|c|}
\hline 1.000& 1.100 &  1.200 & 1.300&  1.400 & 1.500&1.600 & 1.700\\\hline
     5.000 &6.000 & 7.105 & 8.315 & 9.628& 11.045 &12.566 & 14.189\\\hline
     5.000 &6.050&7.200& 8.450&9.800& 11.250&12.800&14.450\\\hline
     0.000&-0.050&-0.095&-0.136& -0.172&-0.205& -0.234&-0.261\\\hline

    \end{tabular}
   % \caption{Caption}
   % \label{tab:my_label}
\end{table}

2)

\begin{table}[H]
    \centering
    \begin{tabular}{|c|c|c|c|c|c|c|c|}
\hline 1.000 &1.100& 1.200&1.300 &1.400& 1.500&1.600 & 1.700\\\hline
     5.000 &6.051  &7.202& 8.453 &9.804 &11.255&12.806& 14.457\\\hline
     5.000& 6.050& 7.200& 8.450 &9.800 &11.250&12.800&14.450\\\hline
     0.000 &0.001 &0.002 &0.003 & 0.004 & 0.005 & 0.006  & 0.007\\\hline

    \end{tabular}
   % \caption{Caption}
   % \label{tab:my_label}
\end{table}

3)

\begin{table}[H]
    \centering
    \begin{tabular}{|c|c|c|c|c|c|c|c|}
\hline 1.000&1.100 &1.200 & 1.300& 1.400 &1.500& 1.600 & 1.700\\\hline
     5.000 &6.053& 7.205&8.457 & 9.809& 11.260&12.812 &14.463\\\hline
     5.000 &6.050&7.200& 8.450 &9.800& 11.250&12.800& 14.450\\\hline
     0.000 & 0.002 & 0.005& 0.007& 0.009 & 0.010& 0.012 & 0.013\\\hline

    \end{tabular}
   % \caption{Caption}
   % \label{tab:my_label}
\end{table}

\subsection{Разностный метод решения краевой задачи
для ОДУ 2-го порядка}

\subsubsection{Код}

\begin{lstlisting}
std::vector<double> differences_method(const int n, 
const double h, const std::function<double(double)>& p,
const std::function<double(double)>& q, 
   const std::function<double(double)>& f)
{
    std::vector<std::vector<double>> mat;
    std::vector<double> xk, fk;
    xk.push_back(0);
    for (size_t i = 1; i <= n; ++i)
    {
        xk.push_back(xk[i - 1] + h);
    }
    for (size_t i = 0; i < xk.size(); ++i)
    {
        fk.push_back(f(xk[i]));
    }
    int y0 = 0, yn = 0;
    for (size_t i = 0; i < n - 2; ++i)
        mat.push_back(std::vector<double>(n - 2));
    for (size_t i = 0; i < n - 2; ++i)
    {
        mat[i][i] = -2. / (h * h) + q(xk[i]);
        if (i != 0)
        {
            mat[i][i - 1] = 1. / (h * h) - p(xk[i]) /(2 * h);
        }
        if (i != n - 3)
        {
            mat[i][i + 1] = 1. / (h * h) + p(xk[i]) /(2 * h);
        }
    }

    Matrix a = Matrix(mat);
    auto yk = Matrix::three_diagonal_matrix(a, fk)[2];
    yk.push_back(0.);
    yk.insert(yk.begin(), 0);
    return yk;
}

void print_differences_method(std::vector<double> Xk, 
std::vector<double> Yk, std::vector<double> Y)
{
    const size_t width = 20;
    std::cout << std::fixed << std::setprecision(6);
    for (size_t i = 0; i < Xk.size(); i++)
    {
        std::cout << std::setw(width) << Xk[i];
    }
    std::cout << "\n";
    for (size_t i = 0; i < Yk.size(); i++)
    {
        std::cout << std::setw(width) << Yk[i];
    }
    std::cout << "\n";
    for (size_t i = 0; i < Y.size(); i++)
    {
        std::cout << std::setw(width) << Y[i];
    }
    std::cout << "\n";
    for (size_t i = 0; i < Yk.size(); i++)
    {
        std::cout << std::setw(width) << Yk[i] - Y[i];
    }
}
\end{lstlisting}

\subsubsection{Вывод}

\begin{table}[H]
    \centering
    \begin{tabular}{|c|c|c|c|c|c|c|c|c|}
\hline 0  &   0.625  & 1.250 &   1.875 & 2.5 &      3.125  &   3.750000  &  4.375  &  5 \\\hline
 0   & 541.605  & 1063.679& 1223.736&   838.756  &        484.769  &   312.843 &135.178 &    0\\\hline
0 & -8.544  &  -29.296 &-54.931 &  -78.125&    -91.552  & -87.890  &  -59.814  & 0 \\\hline
0     &     550.150 &1092.976  & 1278.668 & 916.881  &  576.322  &400.734 &  194.993 &0\\\hline
    \end{tabular}
   % \caption{Caption}
   % \label{tab:my_label}
\end{table}


\subsection{Метод неопределённых коэффициентов решения краевой задачи
для ОДУ 2-го порядка}

Решить методом неопределённых коэффициентов краевую задачу из прошлого задания.


\subsubsection{Код}

\begin{lstlisting}
std::vector<double> unknown_koeff_method(const int n, 
std::vector<double> xk, std::function<double(double)> f, 
std::function<double(double)> p, std::function<double(double)> q, 
std::function<double(double, int)> phi_k,
std::function<double(double, int)> phi_k_derived, 
std::function<double(double, int)> phi_k_derived2)
{
    std::vector<std::vector<double>> mat_ak;
    for (size_t i = 0; i < n; i++)
        mat_ak.push_back(std::vector<double>(n));


    for (size_t j = 0; j < n; j++)
    {
        for (size_t k = 0; k < n; k++)
        {
            mat_ak[j][k] = phi_k_derived2(xk[j], k) + 
            p(xk[j]) * phi_k_derived(xk[j], k) +
            q(xk[j]) * phi_k(xk[j], k);
        }
    }
    Matrix A(mat_ak);


    std::vector<std::vector<double>> fk;
    for (size_t i = 0; i < n; i++)
    {
        fk.push_back(std::vector<double>(1));
    }
    for (size_t i = 0; i < n; i++)
        fk[i][0] = f(xk[i]);
    Matrix B(fk);

    auto temp = Matrix::forward_gauss(A, B);
    auto ak = Matrix::back_gauss(temp);
  
    return ak;
}

//in main function
auto y_new = [&ak, &phi_k](double x)
{
        double ans = 0;
        for (size_t i = 0; i < ak.size(); ++i)
            ans += ak[i] * phi_k(x, i);
        return ans;
};
\end{lstlisting}


\subsubsection{Вывод}


\begin{table}[H]
    \centering
    \begin{tabular}{|c|c|c|c|c|c|c|c|c|}
\hline 0.625 & 1.250  & 1.875 &    2.500  & 3.125 & 3.750  &4.375  &      5.000\\\hline
-704.808  & -1134.035 & -1009.307 & -747.954  & -571.162 &-413.318 &       -250.851  &  0.000\\\hline
-8.545 & -29.297 & -54.932 & -78.125&-91.553&  -87.891&   -59.814 &              0.000\\\hline
-696.263 & -1104.738 & -954.375 & -669.829 & -479.609 & -325.427 &           -191.037   &  0.000\\\hline
    \end{tabular}
   % \caption{Caption}
   % \label{tab:my_label}
\end{table}

Примечание. При увеличении количества внутренних узловых точек погрешность уменььшается (при n=100 погрешность составляет порядка 2.2801)


\section{Численные методы решения интегральных уравнений}

\subsection{Решение интегрального уравнения Фредгольма в случае
вырожденного ядра}

Решить следующее интегральное уравнение с вырожденным ядром:

$$y(x) + 1 * \int_{0}^{1}(x * 5 + x^2t^2 + x^3t^3) * y(t)dt = 5 * (\frac{4}{3}x + \frac{1}{4}x^2 + \frac{1}{5}x^3)$$

Где $y_{correct}(x) = 5x$


\subsubsection{Код}

\begin{lstlisting}
std::vector<double> integral_equation()
{

    std::vector<std::vector<double>> A_mat = {
        {1 + 1. / 3, 1. / 4, 1. / 5},
        {1. / 4, 1 + 1. / 5, 1. / 6},
        {1. / 5, 1. / 6, 1 + 1. / 7}
    };
    Matrix A(A_mat);

    std::vector<std::vector<double>> phi_mat = {
        {20. / 9 + 5. / 16 + 1. / 5},
        {20. / 12, 5. / 20, 1. / 6},
        {20. / 15, 5. / 24, 1. / 7}
    };
    Matrix phi(phi_mat);

    auto tmp = Matrix::forward_gauss(A, phi);
    auto q = Matrix::back_gauss(tmp);


    return q;
}


\\in main function
auto q = integral_equation();
auto y_method = [&f, lambda, &q, &ak, n](double x)
{
    double sum = 0;
    for (size_t i = 0; i < n; ++i)
        sum += q[i] * ak(x, i + 1);
    sum *= lambda;
    return f(x) - sum;
};
\end{lstlisting}

\subsubsection{Вывод}

\begin{table}[H]
    \centering
    \begin{tabular}{|c|c|c|c|c|c|c|c|c|c|}
\hline 0.000  & 0.100   &0.200   & 0.300&0.400 &0.500 & 0.600 &   0.700 & 0.800   & 0.900\\\hline
0.000  & 0.493 & 0.995 & 1.506 & 2.029  & 2.565 & 3.117 &3.684   &            4.271  & 4.876\\\hline
0.000  & 0.500  & 1.000 & 1.500 & 2.000& 2.500 & 3.000&3.500&   4.000  &  4.500\\\hline
0.000&-0.007 &-0.005& 0.006  & 0.029 &0.065&0.117&0.184&0.271&    0.376\\\hline
    \end{tabular}
   % \caption{Caption}
   % \label{tab:my_label}
\end{table}


\subsection{Решение интегрального уравнения Фредгольма квадратурным методом}

Решить следующее интегральное уравнение квадратурным методом:

$$y(x) + 1 * \int_{0}^{1}(x * 5 + x^2t^2 + x^3t^3) * y(t)dt = 5 * (\frac{4}{3}x + \frac{1}{4}x^2 + \frac{1}{5}x^3)$$

Где $y_{correct}(x) = 5x$


\subsubsection{Код}

\begin{lstlisting}
std::vector<double> integral_equation_quadro(std::vector<double> xk,
std::function<double(double, double)> A, 
std::function<double(double)> f, 
double h, double lambda)
{
    std::vector<std::vector<double>> A_vec;
    const size_t n = xk.size();
    for (size_t i = 0; i < n - 1; ++i)
    {
        A_vec.push_back(std::vector<double>(n - 1));
    }


    for (size_t i = 0; i < n-1; ++i)
        for (size_t j = 0; j < n - 1; ++j)
        {
            if (i == j)
            {
                A_vec[i][j] = 1 + lambda * h * A(xk[i], xk[j]);
            }
            else
            {
                A_vec[i][j] = lambda * h * A(xk[i], xk[j]);
            }
        }

    std::vector<std::vector<double>> B_vec(n - 1);
    for (size_t i = 0; i < n - 1; ++i)
    {
        B_vec[i].push_back(f(xk[i]));
    }


    Matrix A_mat(A_vec);
    Matrix B(B_vec);

    auto tmp = Matrix::forward_gauss(A_mat, B);
    return Matrix::back_gauss(tmp);
}
\end{lstlisting}


\subsubsection{Вывод}

\begin{table}[H]
    \centering
    \begin{tabular}{|c|c|c|c|c|c|c|c|c|c|c|c|c|}
\hline 0   &  0.07   & 0.15  &  0.23& 0.30& 0.38&  0.46 &        0.53    & 0.61   &    0.69   &     0.76  & 0.84 & 0.92\\\hline
  0  &  0.41  & 0.83 &1.25&1.68& 2.13 &2.58&3.04&  3.51& 3.99  & 4.49  & 5.00 & \\\hline
  0 & 0.38& 0.76 & 1.15 & 1.53& 1.92& 2.30& 2.69 & 3.07 & 3.46  & 3.84 & 4.23 &  4.61 \\\hline
   0 & 0.02 &  0.06  & 0.10 &0.15 &0.20  & 0.27 &0.34  &       0.43   &   0.53   &   0.64 &  0.77 & \\\hline
    \end{tabular}
   % \caption{Caption}
   % \label{tab:my_label}
\end{table}

\newpage

% Раздел "Заключение"
\conclusion
В настоящей работы приведен пример оформления студенческой работы средствами системы \LaTeX.

Показано, как можно оформить документ в соответствии:
\begin{itemize}
    \item с правилами оформления курсовых и выпускных квалификационных работ, принятых в Саратовском государственном университете в 2012 году;
    \item с правилами оформления титульного листа отчета о прохождении практики в соответствии со стандартом.
\end{itemize}


%Библиографический список, составленный вручную, без использования BibTeX
%
%\begin{thebibliography}{99}
%  \bibitem{Ione} Источник 1.
%  \bibitem{Itwo} Источник 2
%\end{thebibliography}

%Библиографический список, составленный с помощью BibTeX
%
\bibliographystyle{gost780uv}
\bibliography{thesis}

% Окончание основного документа и начало приложений
% Каждая последующая секция документа будет являться приложением
\appendix

\section{Нумеруемые объекты в приложении}

\begin{table}[!ht]
	\footnotesize
	\caption{Results of pass-fail dictionary reduction with the help
	of masks} \label{table-2}
	\begin{tabular}{|p{1.5cm}|
	                 p{1.5cm}|
	                 p{1.5cm}|
	                 p{1.5cm}|
	                 p{1cm}|
	                 p{1.5cm}|
	                 p{1.5cm}|
	                 p{1cm}|}
		\hline \centering Circuit & Number of modelled faults & Number of test
		vectors in the test set & The volume of pass-fail dictionary,
		\linebreak bit & The volume of found mask & The volume of
		masked dictionary, \linebreak bit & \raggedright \% of pass-fail dictionary
		& CPU running time, \linebreak min
		\\
		\hline S298 & 177 & 322 & 56994 & 30 & 5310 & 9,32\% & 0,07\\
		\hline S344 & 240 & 127 & 30480 & 29 & 6960 & 22,83\% & 0,04\\
		\hline S349 & 243 & 134 & 32562 & 35 & 8505 & 26,12\% & 0,05\\
		\hline S382 & 190 & 2074 & 394060 & 28 & 5320 & 1,35\% & 0,43\\
		\hline S386 & 274 & 286 & 78364 & 65 & 17810 & 22,73\% & 0,26\\
		\hline S400 & 194 & 2214 & 429516 & 32 & 6208 & 1,45\% & 0,99\\
		\hline S444 & 191 & 2240 & 427840 & 30 & 5730 & 1,34\% & 0,98\\
		\hline S526 & 138 & 2258 & 311604 & 28 & 3864 & 1,24\% & 0,61\\
		\hline S641 & 345 & 209 & 72105 & 58 & 20010 & 27,75\% & 0,24\\
		\hline S713 & 343 & 173 & 59339 & 58 & 19894 & 33,53\% & 0,19\\
		\hline S820 & 712 & 1115 & 793880 & 147 & 104664 & 13,18\% & 9,09\\
		\hline S832 & 719 & 1137 & 817503 & 151 & 108569 & 13,28\% & 9,20\\
		\hline S953 & 326 & 14 & 4564 & 13 & 4238 & 92,86\% & 0,01\\
		\hline S1423 & 293 & 150 & 43950 & 58 & 16994 & 38,67\% & 0,15\\
		\hline S1488 & 1359 & 1170 & 1590030 & 158 & 214722 & 13,50\% & 26,69\\
		\hline
	\end{tabular}
\end{table}

\begin{equation}
  F(x)=\int\limits_a^bf(x)\,dx.
\end{equation}


\begin{figure}[!ht]
	\centering
	\caption{\label{fig:f3}%
	Подпись к рисунку}
\end{figure}


\begin{table}[!ht]
\caption{}
\begin{tabular}{|c|c|}
  \hline
  0 & 1\cr
  \hline
  1 & 0\cr
  \hline
\end{tabular}
\end{table}


\section{Листинг программы}\label{pril-1}
Код приложения \verb"task.pl".


\end{document}